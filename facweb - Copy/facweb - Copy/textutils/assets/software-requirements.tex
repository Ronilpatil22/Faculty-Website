											SRS(SOFTWARE REQUIREMENT SPECIFICATION)

Contents

1 Introduction                                                                      3
1.1 Purpose . . . . . . . . . . . . . . . . . . . . . . . . . . . . . . . . . . . . . 3
1.2 Intended Audience and Reading Suggestions . . . . . . . . . . . . . . . . . 3
1.3 Project Scope . . . . . . . . . . . . . . . . . . . . . . . . . . . . . . . . . . 3

2 Overall Description                                                               5
2.1 Product Perspective . . . . . . . . . . . . . . . . . . . . . . . . . . . . . . 5
2.2 User Classes and Characteristics . . . . . . . . . . . . . . . . . . . . . . . 5
2.3 Product Functions . . . . . . . . . . . . . . . . . . . . . . . . . . . . . . . 6
2.4 Operating Environment . . . . . . . . . . . . . . . . . . . . . . . . . . . . 7
2.5 Design . . . . . . . . . . . . . . . . . . . . . . . . . . . . . . . . . . . . . . 7

3 System Features                                                                   9
3.1 Description and Priority . . . . . . . . . . . . . . . . . . . . . . . . . . . . 9
3.2 Functional Requirements . . . . . . . . . . . . . . . . . . . . . . . . . . . . 9

4 Other Nonfunctional Requirements                                            10
4.1 Performance Requirements . . . . . . . . . . . . . . . . . . . . . . . . . . 10
4.2 Security Requirements . . . . . . . . . . . . . . . . . . . . . . . . . . . . . 10
4.3 Software Quality Attributes . . . . . . . . . . . . . . . . . . . . . . . . . . 10
4.4 Business Rules . . . . . . . . . . . . . . . . . . . . . . . . . . . . . . . . . 10

5 Other Requirements                                                              11

1 Introduction
1.1 Purpose
Nowadays it very common for people to search on internet for every kinds of things. So why not create a good impression on them by creating a personal website containing information about your qualification, achievements and professional life. This way the person would get to know a lot about you . And this is why this days nearly everyone is trying to create his own website and share it with the whole world, especially for professors. This faculty website website that we are creating for Prof. Puneet gupta would help students get to know sir better professionally.

1.2 Intended Audience and Reading Suggestions
This SRS is for developers, project managers, users and testers. Further the discussion
will provide all the internal, external, functional and also non-functional informations
about "Faculty Website".

1.3 Project Scope
  This Project has a lot of scope in the future as it will help people to look after prof's qualification and achievements and furthermore this project can be extended further for creating a student portal wherein student's currently enrolled in course taught by prof would be able to submit there assignments and download lecture slides.

2 Overall Description
2.1 Product Perspective 
From a product perspective, the website for professor would be designed to effectively showcase the professor's qualifications, achievements, and experience to potential students, colleagues, and employers. The website would be user-friendly, visually appealing, and contain all the relevant information about the professor in a clear and concise manner.

2.2 User Classes and Characteristics
Here are some possible user classes and their associated characteristics:

Students:
Characteristics: Students may be prospective or current students who are interested in learning more about the professor's qualifications, teaching style, and course offerings.
Needs: Students may be looking for information about the professor's academic background, research interests, publications, and teaching philosophy, as well as details about specific courses.

Colleagues:
Characteristics: Colleagues may be other professors or researchers who are interested in collaborating with the professor on research projects or networking opportunities.
Needs: Colleagues may be looking for information about the professor's research interests, publications, and professional affiliations.

Employers:
Characteristics: Employers may be potential employers who are interested in hiring the professor for a teaching or research position.
Needs: Employers may be looking for information about the professor's academic background, research experience, teaching experience, and professional accomplishments.

General public:
Characteristics: The general public may include individuals who are interested in the professor's work or are seeking general information about the professor.
Needs: The general public may be looking for information about the professor's academic background, research interests, publications, and professional accomplishments.

2.3 Product Functions
Here are some product functions that would be included in the website:
1.Profile creation: The website would allow the professor to create a profile that includes their qualifications, achievements, teaching experience, and research interests.

2.Course management: The website would allow prof to manage course materials, such as syllabi, assignments, and grades, and communicate with students.

3.Publication management: The website would allow the professor to manage their publications, such as books, articles, and conference papers, and provide access to them for users.

4.Testimonial management: The website would allow the professor to manage testimonials from former students or colleagues and display them on their profile page.

5.Contact management: The website would allow users to easily contact the professor via email or through a contact form on the website.

6.Multimedia content management: The website would allow the professor to upload and manage multimedia content such as videos, podcasts, or presentations that showcase their work.

7.Social media integration: The website would integrate with the professor's social media accounts, allowing users to follow and share their content on various platforms.

8.Mobile responsiveness: The website would be designed to be mobile-responsive, allowing users to access the site on a variety of devices and screen sizes.

2.4 Operating Environment
Following are the operating requirements for the website:
1.Web server: The website would require a web server that supports the required technologies, such as HTML, CSS, Django and JavaScript.

2.Operating system: The web server should be running on a stable and reliable operating system, such as Linux, Windows Server, or macOS Server.

3.Web browser support: The website would be designed to work on a variety of web browsers, such as Google Chrome, Mozilla Firefox, Apple Safari, and Microsoft Edge.

4.Database management system: The website may require a database management system, such as MySQL or PostgreSQL, to store and manage data related to courses, publications, and user accounts.

5.Content management system: The website will require a content management system, such as WordPress or Drupal, to manage and update content on the site.

6.Security protocols: The website would also implement appropriate security protocols, such as SSL encryption and secure login procedures, to protect user data and prevent unauthorized access.

8.Mobile responsiveness: The website would be designed to be mobile-responsive, allowing users to access the site on a variety of devices and screen sizes.

9.Backup and recovery: The website would also have a backup and recovery plan in place, including regular backups of website data and a disaster recovery plan in case of data loss or website downtime.

2.5 Design
The home page consists of a basic introduction about sir and contains all the social media handle links of the professor in addition to a testimonial section and a FAQ section and also contain a link for signing up for the newsletter.

About Me section will contain information about professor like his qualification,achievements,courses taught by sir,experience,recognition,invited talks and guests.

My Work section will contain articles,citations,books,youtube videos by professor along with future plans of professor and internship opportunities under profesor.

Research Group section will contain information about the research group of professor.

Contact details section will contain contact information of prof.

Student portal section will contain a portal for students currently enrolled in courses taught by professor for submitting there assignments and downloading lecture notes.

Discussion Board will allow any user to post a question to the professor which can be answered by him.

Feedback section will allow users to provide there valuable feedback for the website.

3 System Features 
3.1 Description and Priority
There are many features of this website that has been already described under 2.5 section of this document.
Features that are under priority for this website are the home page section,about me section, my work section and contact details section which seem to be the most important ones considering the purpose of this website. 
After this the next priority would be to create a research groups section and feedback section.
Rest all features are of least priority for this project.

3.2 Functional Requirements 
The ”FACULTY WEBSITE” website is being build on HTML,CSS,JAVASCRIPT,DJANGO
Back-End - DJANGO
Font-End - HTML,CSS, JavaScript.
Back-End - MySQL

4 Other Nonfunctional Requirements  
4.1 Performance Requirements 
Following are the performance requirements that will be fulfilled by the website:
1.Response time: The website would respond quickly to user requests, with an average page load time of no more than 3 seconds.

2.Uptime: The website would have a high uptime rate of at least 99%, meaning it should be available and functional for at least 99% of the time.

3.Scalability: The website would be able to handle increases in traffic, without experiencing significant slowdowns or errors. This may involve using scalable hosting solutions, such as cloud hosting or load balancing.

4.Concurrent user support: The website would be able to handle multiple users accessing the site at the same time, without experiencing significant slowdowns or errors.

5.Browser compatibility: The website would be designed to work with a variety of web browsers and browser versions, without experiencing significant issues or errors.

6.Security: The website would implement appropriate security measures, such as SSL encryption and secure login procedures, to protect user data and prevent unauthorized access.

4.2 Security Requirements 
Following are the security requirements that will be fulfilled by the website(All this are mainly for the student protal section of the website):
1.User authentication: The website would implement secure login procedures, such as two-factor authentication or CAPTCHA, to prevent unauthorized access to user accounts.

2.Password policies: The website would enforce strong password policies, such as requiring a minimum length and complexity, to prevent users from using weak or easily guessable passwords.

3.Data encryption: The website would use SSL encryption to protect user data during transmission, and may also use database encryption to protect sensitive data at rest.

4.Backup and recovery: The website would implement regular backup and recovery procedures, to ensure that data can be recovered in the event of a security breach or data loss.

5.Privacy policy: The website would have a clearly defined privacy policy that outlines how user data is collected, stored, and used, and provides users with options to control their data.

4.3 Software Quality Attributes
Following are the software quality attributes that will be integrated with the website:
1.Usability: The website would be easy to use and navigate, with a clear and intuitive interface that enables users to quickly find the information they need.

2.Reliability: The website would be reliable and consistent in its performance, with minimal downtime or errors.

3.Maintainability: The website would be designed to be easy to maintain and update, with well-organized code and clear documentation that enables developers to quickly identify and fix issues.

4.Performance: The website would be designed to perform well, with fast page load times and minimal server load.

4.4 Business Rules 
"Faculty Website" is mainly for storing information and background about professor like his achievements and qualification and providing him an organised platform for showcasing his work,articles,citations etc.

5 Other Requirements 
"Faculty website" will need constant updation and maintanence and hence we will try to refactor the code as good as possible.